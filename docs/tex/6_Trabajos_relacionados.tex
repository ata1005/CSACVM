\capitulo{6}{Trabajos relacionados}

Hay varias aplicaciones dedicadas a la generación de currículos muy similares a lo que se quiere conseguir con este proyecto,

Una de ellas es, por ejemplo, el Curriculum Vitae Normalizado (CVN) del ministerio. Esta herramienta es el editor de currículos más completo y específico que podemos encontrar y tiene la ventaja de estar normalizado y estandarizado en lo que respecta a la mayoría de empresas y administraciones públicas. 

Al estar normalizado, forman la regla para la mayoría de los currículos que se necesitan en una empresa o una administración pública en España. La desventaja de este es, por ejemplo, que tiene una sobrecarga de datos y campos, muchos de ellos innecesarios, hasta el punto en que se puede hacer realmente tedioso completarlo, más aún a la hora de rellenar un documento para uso genérico. 

Además de esto, esta herramienta se enfoca mucho en la información y experiencia académica y no tanto en lo laboral.

También tenemos un editor más pequeño y más visual, como es \href{https://www.cvmaker.es/}{\mbox{CV Maker}}\footnote{CV Maker Online: https://www.cvmaker.es/}, que comprende los campos principales de un currículum, bastante similar a las características de este proyecto en cuanto a sencillez y objetivos.

Sin embargo, la necesidad de realizar este proyecto, viene dada por las siguientes ventajas:
\begin{itemize}
 \item Se necesita un gestor interno y común para la empresa, para que sus usuarios (trabajadores) tengan un currículum con un formato similar en cuanto a estructura y datos.
 \item Es muy útil para la empresa al tener formatos únicos de los currículos y por los que poder filtrar a través de distintos campos.
 \item Es una herramienta sencilla de instalar y mantener con el tiempo, así como para desarrollar nuevas funciones y utilidades a futuro.
 \item El hardware y software utilizado es idéntico al resto de proyectos de la empresa, por lo que es muy fácil integrarlo con los demás.
\end{itemize}
