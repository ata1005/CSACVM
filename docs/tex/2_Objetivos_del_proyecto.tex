\capitulo{2}{Objetivos del proyecto}
CSACVM es una aplicación web con formato de red social dedicada principalmente a la gestión de currículos vitae por parte de una empresa, en este caso CSA (Centro de Servicios Avanzados).

El objetivo de esta aplicación se puede dividir en dos apartados. Por un lado se describe el propósito general de la funcionalidad de la aplicación en el contexto de la empresa, y por otro se describen las técnicas utilizadas para lograr estos objetivos.

\section{Objetivos generales}
\begin{itemize}
 \item Dotar a la aplicación de un estilo de red social intuitivo y ampliable a más funcionalidad en el futuro.
 \item Gestión de currículos por parte de cada usuario/trabajador de la empresa, que pueda añadir nuevas entradas para mantener actualizado el documento y exportarlo a \emph{PDF} cuando se desee.
 \item Mantenimiento principal de usuarios (perfiles, notas, configuración) como en cualquier red social.
 \item Administración y gestión de usuarios y currículos por parte del personal de Recursos Humanos de la empresa para poder explotar los datos.
\end{itemize}

\section{Objetivos técnicos}
\begin{itemize}
 \item Desarrollo de la aplicación en Visual Studio 2022 con .NET 6.0 y Core como herramienta de gestión de modelos.
 \item Aplicar un patrón de arquitectura como el Modelo Vista-Controlador (MVC).
 \item Gestión de base de datos gracias a SQL Management Studio y SQL Server.
 \item Uso del \emph{code-first} como método para unir el modelo con la base de datos (a través de migraciones en Visual Studio con \emph{.NET}).
 \item Descarga de dependencias en Visual Studio que proporcionan librerías para el desarrollo de diversas funcionalidades del proyecto.
 \item Despliegue de la aplicación a través del uso de IIS Express.
 \item Uso de Github para alojar el repositorio del proyecto.
 \item Manejo y gestión de las diferentes fases del proyecto a través de las \emph{issues} y \emph{milestones} que proporciona Github.
 \item Utilizar una máquina virtual para poder alojar el servidor web.
\end{itemize}