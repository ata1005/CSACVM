\capitulo{1}{Introducción}

En esta parte de la documentación se describirá el contexto de la aplicación en base a qué
problemas existen, cómo surge la idea y cómo se van a solucionar dichos problemas. Asimismo, 
se describe la estructura de la memoria y los elementos que se adjuntan.

\section{Contexto}
Cada vez que queremos solicitar un puesto en una empresa, debemos tener 
siempre actualizado nuestro currículum. Esto puede ser muy molesto en el caso
de que se haga a mano, y varias aplicaciones que se pueden encontrar por Internet 
no resultan del todo cómodas para esto.

Desde el punto de vista de la empresa es todavía peor, pues cada currículum tiene un 
formato y estructura diferentes y es tedioso a la hora de filtrarlos por los trabajadores
de la empresa.

Y esto ya no solo es un problema para los currículos que vienen de nuevos trabajadores,
sino que también lo es para los que llevan un determinado tiempo en la empresa, ya que,
al estar hechos a mano, pueden estar años sin actualizarse.

El principal problema que se aborda en esta aplicación es el resultante de todo lo mencionado
previamente: se necesita un gestor interno que unifique la estructura y el formato
de, no solo los currículos de nuevos trabajadores, sino de aquellos que lleven tiempo
en la empresa y, además, que estos últimos se mantengan actualizados con el tiempo.

\section{Estructura del TFG}

La documentación entregada se divide en dos partes: la memoria y los anexos.
La memoria comprende este documento y tiene la siguiente estructura:

\begin{itemize}
\item
  \textbf{Introducción:} breve descripción del problema a resolver y la
  solución propuesta. Estructura de la memoria y listado de materiales
  adjuntos.
\item
  \textbf{Objetivos del proyecto:} las diferentes metas que se quieren 
  alcanzar a través del proyecto.
\item
  \textbf{Conceptos teóricos:} explicaciones teóricas sobre el funcionamiento de varias
  de las herramientas que se utilizan, como librerías y dependencias.
\item
  \textbf{Técnicas y herramientas:} listado de las diferentes utilidades
  que se usan en el desarrollo del proyecto, tales el como entorno, lenguajes,
  librerías, etc.
\item
  \textbf{Aspectos relevantes del desarrollo:} aspectos a destacar durante el
  desarrollo de la aplicación que puedan tener cierto interés informativo.
\item
  \textbf{Trabajos relacionados:} estudio de otras aplicaciones o herramientas similares
  a la que se presenta y sus diferencias, así como un análisis de necesidad del desarrollo de
  esta aplicación en función de las buscadas.
\item
  \textbf{Conclusiones y líneas de trabajo futuras:} conclusiones que se obtienen
  al finalizar el proyecto y posibles mejoras a futuro.
\end{itemize}

Los anexos, a su vez, tienen la siguiente estructura:
\begin{itemize}
\item
  \textbf{Plan del proyecto:} planificación de las diferentes fases que abarcan 
  el proyecto y estudio de la viabilidad del mismo.
\item
  \textbf{Especificación de requisitos del software:} se describe qué es necesario
  hacer en cuanto a estructura y funcionalidad del proyecto, descripción del problema
  y alcance de los requisitos.
\item
  \textbf{Especificación de diseño:} se describe el diseño de las diferentes partes
  que forman la estructura del proyecto, tales como la base de datos, la arquitectura,
  páginas, etc.
\item
  \textbf{Manual del programador:} manual técnico que recoge aspectos importantes del 
  código del proyecto y de la instalación del servidor.
\item
  \textbf{Manual de usuario:} manual orientado a la guía de la aplicación para los
  usuarios que quieran utilizarla en cuanto a funcionamiento y comprensión de las distintas
  partes de la aplicación.
\end{itemize}

\section{Otros materiales}
Junto a la memoria y los anexos, también se adjuntan otros materiales. En este caso 
se entrega el más importante para las pruebas del proyecto, que es la máquina virtual.

La máquina virtual se entrega en forma de ``.ova'' y tiene las siguientes partes:

\begin{itemize}
\item
	SQL Server con la base de datos activa.
\item
	IIS Express configurado con la aplicación.
\item	
	Carpeta de guardado de la aplicación.
\item	
	Servidor web lanzado en localhost.
\end{itemize}

Esta máquina virtual estará accesible a través del repositorio de GitHub\footnote{Url del proyecto: https://github.com/ata1005/CSACVM} (\href{https://github.com/ata1005/CSACVM}{CSACVM - Git}) y a través del USB que se entregue directamente a la Universidad de Burgos. 