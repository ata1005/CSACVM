\capitulo{7}{Conclusiones y Líneas de trabajo futuras}

En esta última sección de la memoria se expondrán tanto las conclusiones extraídas a raíz de la finalización del proyecto, como las conclusiones de análisis de futuras mejoras o cosas que podrían cambiarse de la aplicación.

\section{Conclusiones}

En cuanto a las conclusiones que podemos extraer, se destacan:

\begin{itemize}
\item
  La aplicación cumple el objetivo inicial en cuanto a sencillez, comodidad y efectividad.
  Los usuarios tanto de la empresa como nuevos trabajadores podrán dejar y mantener su currículum actualizado y los administradores y trabajadores de recursos humanos podrán explotar la información de forma interna y utilizar la herramienta para los fines que se deseen.
\item
  El desarrollo del proyecto con las herramientas utilizadas ha sido una decisión acertada, ya que no solo es una de las formas más efectivas para desarrollar este tipo de aplicaciones, sino que además el haberla utilizado previamente ha hecho que, en muchos casos, evite caer en ciertos errores y esto se vea afectado negativamente en el tiempo del desarrollo.
\item
  Gracias a este proyecto en el que se utilizan todo tipo de librerías, dependencias y herramientas, he logrado no solo profundizar en aquellas que ya conocía de antes, sino además descubrir nuevas formas de desarrollo y competencias a raíz de metodologías que con las que no estaba familiarizado. En este sentido también destaco crear el entorno virtual por mí mismo y dotarlo de todo lo necesario para hacer un despliegue correcto.
\item
  La metodología que se ha aplicado me ha ayudado tanto a evitar fallos de forma temprana como a reducir el tiempo de desarrollo de muchas tareas, ya que, como se ha explicado en el cuarto apartado, consigo priorizar las fases en tiempo y en necesidad de desarrollo.
\end{itemize}

\section{Líneas de trabajo futuras}

La entrega de este proyecto de fin de grado no marca el punto y final de la aplicación, sino el primer despliegue de la versión oficial de producción en la empresa, que es al final el objetivo principal que se tenía con ella.

A pesar de ser la versión oficial y final de este trabajo, el proyecto se ampliará y estará sujeto a muchos cambios y mejoras que ya se plantean:

\begin{itemize}
\item  En el repositorio del proyecto, se encuentran algunas de las mejoras planteadas para el        desarrollo a futuro de la aplicación, como viene a ser actualizaciones de diseño, nuevas 
  pantallas, atributos de una red social, adición de nuevas funcionalidades similares a la gestión de currículos, poder unir la aplicación con otras internas de la empresa, etc.
\item
  En cuanto a la gestión de currículum, se prevee que a medida que se vaya utilizando, salgan nuevos campos por los que poder filtrar la información, la posibilidad de descargar varios archivos de golpe, añadir nuevos campos para rellenar los currículos, etc.
\end{itemize}
